\usepackage{times} % Times new roman font asked in the guideline
%%%%%%%%%%%%%%% Maths symbols %%%%%%%%%%%%%%
\usepackage{amsfonts}
\usepackage{amsmath}
\usepackage{amsthm}
\usepackage{amssymb}
%\usepackage{siunitx}
%%%%%%%%%%%%%%%%%%%% Figures, tables and captions %%%%%%%%
\usepackage{graphicx}
%\usepackage{lscape}
\usepackage{caption} % Changes font of captions by putting [font=sf] before {caption} if required.
%\usepackage{dpfloat} % Ability to place figure on even or odd page using \begin{leftfullpage}
% \usepackage{longtable}
\usepackage{booktabs}
\usepackage{longtable} %for list of acronyms
\usepackage{threeparttable}
\usepackage{subcaption}
\graphicspath{{Figures/}}
%%%%%%%%%%%%%%%%%%%%%% Track review changes %%%%%%%%%%%%%%%
%\usepackage{xcolor}
%\newcommand{\RV}[1]{\textcolor{violet}{#1}}

%%%%%%%%%%%%%%%%%% Numbering lines in verbatim %%%%%%%%%%%%
%\usepackage{fancyvrb}
%%%%%%%%%%%%%%%%%%%%%%%% Appendices %%%%%%%%%%%%%%%%%%%
%\usepackage{appendix}
%%%%%%%%%%%%%%%%%%%%%%%%%% Navigation %%%%%%%%%%%%%%%%%%%
% \usepackage[hidelinks,colorlinks,allcolors=blue]{hyperref} %soft-copy with color hyperlinks
\usepackage[hidelinks]{hyperref} %hard-copy no color hyperlinks

\renewcommand{\sectionautorefname}{Section}	%captitalize first s in Section while using autoref
\renewcommand{\subsectionautorefname}{Subsection}
\renewcommand{\chapterautorefname}{Chapter}
\renewcommand{\itemautorefname}{Item}
\usepackage{bookmark}	%to use startatroot feature
\newcommand*{\fullref}[1]{\hyperref[{#1}]{\autoref*{#1} \nameref*{#1}}} %navigating titles along with numbers

%%%%%%%%%%%%%%%%%% OPTIONAL %%%%%%%%%%%%%%%%%%
%%%%%%%%%%%%%%%%%% Bengali font for few texts and Russian for a specific bibitem %%%%%%%%%%%%%%%%%%%
%%%%%%%%%%%%%%%%%% XeLaTeX compiler with this setting
% \usepackage{polyglossia}
% \setmainlanguage{english}
% \setmainfont{Times New Roman} % Times new roman font asked in the guideline
% \setotherlanguages{bengali, russian}
% \newfontfamily \bn[Scale=0.9,Script=Bengali]{Kalpurush.ttf} % if you want to use this font, you need to place the font file in the root directory
% \newfontfamily \cyrillicfont[Script=Cyrillic]{STIX2Text-Regular.otf} % if you want to use this font, you need to place the font file in the root directory
% \usepackage{tipa}
%%%%%%%%%%%%%%%% References %%%%%%%%%%%%%%%%%%
\usepackage[dashed=false, backend=biber, style=ieee, natbib=true, maxcitenames=2, mincitenames=1, defernumbers=true]{biblatex}
\usepackage{csquotes}
\bibliography{all_references.bib}
\setcounter{biburlnumpenalty}{7000} %to break DOI having strings of numbers
\usepackage[nospacearound]{extdash} %break already dashed word Otolaryngology in a reference
%%%%%%%%%%%%%% Aesthetic, control of margins, headers, etc. %%%%%%%%%%%%%%%
\usepackage{geometry}
\geometry{
left=30mm,
top=30mm,
right=25mm,
bottom=25mm
}
\setlength{\headheight}{14.49998pt} % fancyhdr need space on top
\usepackage{fancyhdr,layout}
\usepackage{etoolbox}		%to use appto command
%\oddsidemargin 1.6 cm
%\evensidemargin 0.4 cm
\pagestyle{fancy}
\renewcommand{\chaptermark}[1]{\markboth{\MakeUppercase{\chaptername}\ \thechapter.\ #1}{}}		%changing the chaptermark
%\renewcommand{\sectionmark}[1]{\markright{\thesection\ #1}} %not used
\fancypagestyle{main}{%		%pagestyles for mainmatter
\fancyhf{}
% \fancyhead[LE,RO]{\bfseries\thepage}	%page number at HeaderCenter for both even and odd pages % if I make it for two-sided book-like printing
\fancyhead[R]{\bfseries\thepage}	%page number at HeaderCenter for both even and odd pages % If I make it for single-sided printing
%\fancyhead[LO]{\bfseries\rightmark}	%left of odd pages: sectionmark
% \fancyhead[LO]{\bfseries\leftmark}	%left of odd pages: chaptermark % if I make it for two-sided book-like printing
% \fancyhead[L]{\bfseries\leftmark}	%left of odd pages: chaptermark % If I make it for single-sided printing
%\fancyhead[RE]{\bfseries\leftmark} %right of even page: chaptermark %no use in one-side printing
\renewcommand{\headrulewidth}{0pt}	%width of the header rule
\renewcommand{\footrulewidth}{0pt}
}
%\addtolength{\headheight}{14.5pt}
\setlength{\footskip}{0in}
\renewcommand{\footruleskip}{0pt}
\fancypagestyle{plain}{%		%redefining plain page style for first few pages
\fancyhf{}
% \fancyhead[LE,RO]{\bfseries\thepage} % if I make it for two-sided book-like printing
\fancyhead[R]{\bfseries\thepage} % If I make it for single-sided printing
\renewcommand{\headrulewidth}{0pt}
}
\appto\frontmatter{\pagestyle{plain}}
\appto\mainmatter{\pagestyle{main}}
%%%%%%%%%%%%%%%%%%%%%%%%%%% Others %%%%%%%%%%%%%%%%%%%%%
\usepackage{enumerate}
\usepackage{array}	%to use arraybackslash in table
\usepackage{calc}		%to add two length variable
%\usepackage{anyfontsize}

%%%%%%%%%%%%%%%%%%%%%% Control of chapter/section headings according to the guidelines %%%%%%%%%%%%%%%%%%%%%
\usepackage[explicit]{titlesec}

%change the chapter number to Roman only after CHAPTER. But the below first command also changes the chapter number everywhere including section, figures, etc. numbering. So, section, figure, etc numbers are kept at arabic again
\renewcommand{\thechapter}{\Roman{chapter}}
\renewcommand{\thesection}{\arabic{chapter}.\arabic{section}}
\renewcommand{\thefigure}{\arabic{chapter}.\arabic{figure}}
\renewcommand{\thetable}{\arabic{chapter}.\arabic{table}}
\renewcommand{\theequation}{\arabic{chapter}.\arabic{equation}}

\titleformat{\chapter}[display]{\normalfont\bfseries\filcenter}{\MakeUppercase\chaptertitlename\  \thechapter}{24pt}{{#1}}	%CHAPTER with a space
% \titleformat{\chapter}[display]{\large\bfseries\filcenter}{\MakeUppercase\chaptertitlename\ \thechapter}{20pt}{{#1}} % large font size rather than what mentioned in guideline
%\titlespacing*{\chapter}{0pt}{0pt}{24pt}  %KUET guideline. The command controls vertical margins on title; {left}{before-sep}{after-sep}
\titlespacing*{\chapter} {0pt}{0pt}{40pt} % slightly changed to look better

\titleformat{\section}{\normalfont\bfseries}{\thesection}{0.65em}{#1}
%\titlespacing*{\section}{0pt}{24pt}{12pt} 
\titlespacing*{\section} {0pt}{3.5ex plus 1ex minus .2ex}{1ex plus .2ex}	

\titleformat{\subsection}{\normalfont\bfseries}{\thesubsection}{0.65em}{#1}
%\titlespacing*{\subsection}{0pt}{20pt}{10pt}
\titlespacing*{\subsection} {0pt}{3.25ex plus 1ex minus .2ex}{.5ex plus .2ex}

\titleformat{\subsubsection}{\normalfont\bfseries}{\thesubsubsection}{0.65em}{#1}
\titlespacing*{\subsubsection}{0pt}{3.25ex plus 1ex minus .2ex}{.5ex plus .2ex}

%standard spacings from titlesec doc
%\titlespacing*{\chapter} {0pt}{50pt}{40pt}
%\titlespacing*{\section} {0pt}{3.5ex plus 1ex minus .2ex}{2.3ex plus .2ex}
%\titlespacing*{\subsection} {0pt}{3.25ex plus 1ex minus .2ex}{1.5ex plus .2ex}
%\titlespacing*{\subsubsection}{0pt}{3.25ex plus 1ex minus .2ex}{1.5ex plus .2ex}
%\titlespacing*{\paragraph} {0pt}{3.25ex plus 1ex minus .2ex}{1em}
%\titlespacing*{\subparagraph} {\parindent}{3.25ex plus 1ex minus .2ex}{1em}

%%%%%%%%%%%%%%%%%%%%%%% Line Spacing %%%%%%%%%%%%%%%%%%
% spacing 1.5
%\linespread{1.5} 
%\renewcommand{\baselinestretch}{1.5}		%controlling linespacing useing baselinestretch changing, similar to the linespread command. It changes the spacing for everything in the document, including footnotes and tables
%\newcommand{\spacingR}[1]{\linespread{#1}\selectfont}		%different linespacing within the document if linespread is used
\usepackage{setspace}
%\captionsetup{font={stretch=1.5}}	%to address: setspace doesn't change the figure/table captions
\setstretch{1.5}	%onehalfspacing of LaTeX doesn't match with MS Word's 1.5 spacing, so tuned this value. Probably MS Word's 1.5 spacing is equivalent to (slightly less than) doublespaing in LaTeX
%\onehalfspacing
%%%%%%%%%%%%% control the style of toc, lof, lot %%%%%%%%%%%%%%%%
\usepackage{tocloft} 
%making toc centered
\renewcommand{\cfttoctitlefont}{\hfill\large\bfseries}
\renewcommand{\cftaftertoctitle}{\hfill} 
\addtocontents{toc}{~\hfill\textbf{Page}\par}	%to pring the word "Page" above the page numbers
%making lot centered
\renewcommand{\cftlottitlefont}{\hfill\large\bfseries\MakeUppercase}
\renewcommand{\cftafterlottitle}{\hfill} 
%making lof centered
\renewcommand{\cftloftitlefont}{\hfill\large\bfseries\MakeUppercase}
\renewcommand{\cftafterloftitle}{\hfill} 

\renewcommand{\cftpnumalign}{c} %center align the page numbers
\cftsetpnumwidth{1.9em}	%so that the alighment looks good with the title "Page"
%write CHAPTER in front of chapter
\renewcommand{\cftchapfont}{}		%removing boldface
\renewcommand{\cftchappagefont}{} 	%removing boldface from page numbers
\renewcommand{\cftchappresnum}{\bfseries CHAPTER } 
\renewcommand{\cftchapaftersnum}{\quad}	%some space after chapter number
\newlength{\mylen} % a "scratch" length
\settowidth{\mylen}{\bfseries\cftchappresnum\cftchapaftersnum} % extra space
\addtolength{\cftchapnumwidth}{\mylen} % add the extra space

\cftsetindents{section}{\cftchapnumwidth}{\cftsecnumwidth} %pushing extra indent to put right below chapter title. Extra indents equal already calculated extra space (i.e., cftchapnumwidth) created by "CHAPTER " word. Section numwidth default value as cftsecnumwidth kept unaltered.
\newlength{\mysecondlen}
\setlength{\mysecondlen}{\cftchapnumwidth+\cftsecnumwidth} %subsection indent=chapter numwidth plus section numwidth
\cftsetindents{subsection}{\mysecondlen}{\cftsubsecnumwidth} %indentation for subsection just like section

% as per guideline, no dots linking to page number
\renewcommand{\cftsecdotsep}{\cftnodots} %section
\renewcommand{\cftsubsecdotsep}{\cftnodots} %subsection
\renewcommand{\cftfigdotsep}{\cftnodots} %figure; looks weird for multi-line captions
\renewcommand{\cfttabdotsep}{\cftnodots} %table; looks weird for multi-line captions

%formatting frontmatter contents
\renewcommand{\cftpartfont}{}		%removing boldface
\renewcommand{\cftpartpagefont}{\normalfont} 	%removing boldface from page numbers and giving normalfont
\setlength{\cftbeforepartskip}{1.5mm} %reducing vertical space in front matters of contents
\cftsetindents{part}{3em}{\cftpartnumwidth} %extra indentation as per guideline

%%%%%%%%%%%%%%%%%%%%%% Paragraph Indentation %%%%%%%%%%%%%%%
\usepackage{parskip}	%to have a vartical space instead of indentation in paragraph... as per guideline. It has a slight clash with the tocloft package; suggested to declare after toclof